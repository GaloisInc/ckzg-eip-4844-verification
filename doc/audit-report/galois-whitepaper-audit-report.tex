\documentclass[12pt]{galois-whitepaper}
\usepackage{listings}
\usepackage{float}
\usepackage{xspace}
\usepackage{color}
\usepackage{tikz}
\usepackage{url}
\usepackage{amsmath}
\usepackage{amscd}
\usepackage{verbatim}
\usepackage{fancyvrb}
\usepackage{hyperref}
\usepackage{multirow}
\let\verbatiminput=\verbatimtabinput
\VerbatimFootnotes
\DefineVerbatimEnvironment{code}{Verbatim}{}
\DefineVerbatimEnvironment{pseudoCode}{Verbatim}{}
%\hyphenation{Saw-Script}
%\newcommand{\sawScript}{{\sc SawScript}\xspace}
\renewcommand{\textfraction}{0.05}
\renewcommand{\topfraction}{0.95}
\renewcommand{\bottomfraction}{0.95}
\renewcommand{\floatpagefraction}{0.35}
\setcounter{totalnumber}{5}
\definecolor{MyGray}{rgb}{0.9,0.9,0.9}
\makeatletter\newenvironment{graybox}{%
   \begin{lrbox}{\@tempboxa}\begin{minipage}{\columnwidth}}{\end{minipage}\end{lrbox}%
   \colorbox{MyGray}{\usebox{\@tempboxa}}
}\makeatother

\setlength{\parskip}{0.6em}
\setlength{\abovecaptionskip}{0.5em}

\lstset{
         basicstyle=\footnotesize\ttfamily, % Standardschrift
         %numbers=left,               % Ort der Zeilennummern
         numberstyle=\tiny,          % Stil der Zeilennummern
         %stepnumber=2,               % Abstand zwischen den Zeilennummern
         numbersep=5pt,              % Abstand der Nummern zum Text
         tabsize=2,                  % Groesse von Tabs
         extendedchars=true,         %
         breaklines=true,            % Zeilen werden Umgebrochen
         keywordstyle=\color{red},
                frame=lrtb,         % left, right, top, bottom frames.
 %        keywordstyle=[1]\textbf,    % Stil der Keywords
 %        keywordstyle=[2]\textbf,    %
 %        keywordstyle=[3]\textbf,    %
 %        keywordstyle=[4]\textbf,   \sqrt{\sqrt{}} %
         stringstyle=\color{white}\ttfamily, % Farbe der String
         showspaces=false,           % Leerzeichen anzeigen ?
         showtabs=false,             % Tabs anzeigen ?
         xleftmargin=10pt, % was 17
         xrightmargin=5pt,
         framexleftmargin=5pt, % was 17
         framexrightmargin=-1pt, % was 5pt
         framexbottommargin=4pt,
         %backgroundcolor=\color{lightgray},
         showstringspaces=false      % Leerzeichen in Strings anzeigen ?
}

\author{Brett Decker \texttt{decker@galois.com}\\
Ryan McCleeary \texttt{mccleeary@galois.com}\\
Roberto Saltini \texttt{roberto.saltini@consensys.net}\\
Thanh-Hai Tran \texttt{thanh-hai.tran@consensys.net}
}

\title{Formal Verification of the c-kzg library: Establishing the Basis}

\date{August 22, 2024}
\begin{document}
\maketitle

\vspace*{2cm}
%\begin{abstract}
%\end{abstract}

\newpage
\tableofcontents
\newpage

\section{Audit Report}
The goal of this audit report is to determine the adherence of the structure of the c-kzg implementation to the structure of its stated reference,
the \href{https://github.com/ethereum/consensus-specs/blob/dev/specs/deneb/polynomial-commitments.md}{Polynomial Commitments section of the Python Deneb consensus specification} (Python KZG Specification, hereafter) which encodes the \href{https://www.iacr.org/archive/asiacrypt2010/6477178/6477178.pdf}{KZG commitment scheme}.

Identifying similarities and possible discrepancies between the two structures is key to the next two phases of this project.
This comparison will guide us in structuring the Cryptol version of the Python KZG Specification in a way that enables subsequent verification against both the Python KZG specification itself and the c-kzg implementation.

It is important to note that determining the correctness of the c-kzg implementation with respect to the Python KZG Specification is beyond the scope of this document.

Moreover, determining the accuracy of the Python KZG Specification against the pseudocodes in the KZG paper is outside the scope of this project.

\subsection{Audit of the Python KZG Specification against the algorithm in the KZG paper}

The Python KZG Specification follows the ideas in the polynomial commitment scheme $\textsf{PolyCommit}_{\textsf{DL}}$ introduced in the KZG paper to commit data blobs. 
However, the Python KZG Specification has modifications for optimization, and the differences include:
\begin{enumerate}
    \item Instead of using coefficient vectors, polynomials in the Python KZG Specification are represented in evaluation form, i.e., a vector of evaluations of the polynomial at the various \texttt{FIELD\_ELEMENTS\_PER\_BLOB}(4096)-th roots of unity.
    Blobs are of Python Blob type and can be transformed to polynomials in evaluation form by calling the Python function \href{https://github.com/ethereum/consensus-specs/blob/dev/specs/deneb/polynomial-commitments.md\#blob_to_polynomial}{\texttt{blob\_to\_polynomial}}.

    \item Polynomial evaluation and operations are also specified for the evaluation form, e.g., by using the Barycentric evaluation.
    
    \item $\textsf{PolyCommit}_{\textsf{DL}}$ consists of six algorithms: \textsf{Setup}, \textsf{Commit}, \textsf{Open}, \textsf{VerifyPoly}, \textsf{CreateWitness}, and \textsf{VerifyEval}.
    
    \begin{enumerate}
    
    \item \textsf{Setup} is not formalized in the Python KZG Specification.
    The trusted setup is part of the \href{https://github.com/ethereum/consensus-specs/blob/dev/specs/deneb/polynomial-commitments.md#verify_kzg_proof_batch}{\texttt{preset}} 
    and the setup information is stored in constants: 
    \texttt{KZG\_SETUP\_G1}, 
    \texttt{KZG\_SETUP\_G2\_LENGTH}, 
    \texttt{KZG\_SETUP\_G2}, 
    and \texttt{KZG\_SETUP\_LAGRANGE}.
    
    \item \textsf{Commit} is to output a commitment of a polynomial and potentially decommitment information to be used by the \textsf{Open} algorithm.
    In the Python KZG Specification, 
    the commitment of a blob is computed by the Python function \href{https://github.com/ethereum/consensus-specs/blob/dev/specs/deneb/polynomial-commitments.md\#blob_to_kzg_commitment}{blob\_to\_kzg\_commitment}.
    No decommitment information is output by the Python KZG Specification.
    
    \item \textsf{Open} is not included in the Python KZG Specification.
    
    \item \textsf{VerifyPoly} is not also included in the Python KZG Specification.
    
    \item \textsf{CreateWitness} is to output the KZG proof for the evaluation of a polynomial at a specific point.
    It takes as input an evaluation point and a polynomial.
    It outputs a triple comprising the evaluation point, the evaluation of the polynomial at this point and the so called witness. 
    The direct equivalent of  \textsf{CreateWitness} in the Python KZG Specification is the function  \href{https://github.com/ethereum/consensus-specs/blob/dev/specs/deneb/polynomial-commitments.md\#compute_kzg_proof_impl}{compute\_kzg\_proof\_impl}.
    A minor difference is that the proof output by \href{https://github.com/ethereum/consensus-specs/blob/dev/specs/deneb/polynomial-commitments.md\#compute_kzg_proof_impl}{compute\_kzg\_proof\_impl}, compared to the one output by \textsf{CreateWitness}, does not include the evaluation point provided as input.
    Also, in terms of terminology, the Python KZG Specification uses \texttt{proof} to refer to the witness part of the triple.

    The Python KZG Specification also provides the function \href{https://github.com/ethereum/consensus-specs/blob/dev/specs/deneb/polynomial-commitments.md\#compute_kzg_proof}{\texttt{compute\_kzg\_proof}} which takes as input a blob rather than a polynomial.
    
    Additianally, the Python KZG Specification provides the function
    \href{https://github.com/ethereum/consensus-specs/blob/dev/specs/deneb/polynomial-commitments.md\#compute_blob_kzg_proof}{\texttt{compute\_blob\_kzg\_proof}}.
    This function applies the Fiat-Shamir transformation and the hash function to randomly compute a point called \texttt{evaluation\_challenge}, and then computes the KZG proof at this point.
    Such proof only includes the witness.

    Note that all of these functions deal with polynomials in evaluation form.
    
    \item \textsf{VerifyEval} is the verification algorithm.
    It takes as input a commitment $\mathcal{C}$, an evaluation point $z$, the evaluation $\phi(z)$ of a polynomial $\phi$ and a witness.
    It outputs whether the polynomial with commitment $\mathcal{C}$ evaluates to $\phi(z)$ on input $z$.

    The direct equivalent of  \textsf{CreateWitness} in the Python KZG Specification is the function \href{https://github.com/ethereum/consensus-specs/blob/dev/specs/deneb/polynomial-commitments.md\#verify_kzg_proof_impl}{verify\_kzg\_proof\_impl}.
    
    The Python KZG Specification also provides the functions \href{https://github.com/ethereum/c-kzg-4844/blob/main/src/eip4844/eip4844.c\#L551}{\texttt{verify\_blob\_kzg\_proof}} where, in the input parameters of the function, the evaluation point and the evaluation are replaced by a blob.

    \sloppy{In addition, the Python KZG Specification also provides the functions \href{https://github.com/ethereum/consensus-specs/blob/dev/specs/deneb/polynomial-commitments.md\#verify_kzg_proof_batch}{\texttt{verify\_kzg\_proof\_batch}}
    and \href{https://github.com/ethereum/consensus-specs/blob/dev/specs/deneb/polynomial-commitments.md\#verify_blob_kzg_proof_batch}{\texttt{verify\_blob\_kzg\_proof\_batch}}
    to verify multiple KZG proofs.} 
    \end{enumerate}
        
    % \item The commitment scheme $\textsf{PolyCommit}_{\textsf{DL}}$ is designed with numbers in mathematics in mind, but the Python KZG Specification is with Python types, e.g., \texttt{uint256}.

    \item In the KZG paper, the authors use symmetric pairings; however, the BLS12-381 asymmetric pairing is applied in the Python KZG Specification.
\end{enumerate}

\subsection{Audit of the c-kzg Implementation against the Python KZG Specification}

The Python KZG Specification entails 28 functions from four categories: 
Bit-reversal permutation, BLS12-381 helpers, Polynomials, and KZG. 
Table \ref{table:c-kzg-vs-python} identifies all of these functions, according to their category, 
with the GitHub references to Python KZG Specification and the c-kzg implementation. 

In most cases the names of the functions in the Python KZG Specification match the c-kzg implementation. 
The differences include:
\begin{enumerate}
    \item The Python function \texttt{multi\_exp} is represented in the c-kzg implementaion 
by splitting multiplication into two functions, \texttt{g1\_mul} and \texttt{g2\_mul}, to have separate functions 
handle multiplication for the two groups, G1 and G2, separately, followed by a call to \texttt{blst\_doub};
    \item A naming convention difference between the Python function \texttt{bls\_field\_to\_bytes} and the c-kzg 
implementation function \texttt{bytes\_from\_bls\_field};
    \item Two instances where the c-kzg implementation provides two implementing functions (for speed reasons) with 
\texttt{g1\_lincomb\_naive} and \texttt{g1\_lincomb\_fast} for Python function \texttt{g1\_lincomb}, and 
similarly in the case of functions \texttt{blst\_fr\_inverse} and \texttt{blst\_fr\_eucl\_inverse} for Python function 
\texttt{blst\_modular\_inverse} (though there appears currently to be no difference between these two BLST imported 
functions as defined in the 
\href{https://github.com/supranational/blst/blob/415d4f0e2347a794091836a3065206edfd9c72f3/src/recip.c#L135}{BLST repository}). 
    \item One instance where the c-kzg implementaion has the prefix \texttt{fr\_}, for \texttt{fr\_div}
\end{enumerate}

In addition, there are a couple of other slight differences. The computation specified in the Python function 
\texttt{compute\_quotient\_eval\_within\_domain} is handled inside the c-kzg implementation of the 
function \texttt{compute\_kzg\_proof\_impl} instead of as a subroutine.

Another difference is that the function \texttt{blst\_modular\_inverse} uses the builtin Python function 
\href{https://docs.python.org/3/library/functions.html#pow}{\texttt{pow}}, whereas the corresponding BLST functions, as implented in 
\href{https://github.com/supranational/blst/blob/415d4f0e2347a794091836a3065206edfd9c72f3/src/recip.c#L122}{\texttt{reciprocal\_fr}}, 
use BLST utility functions \href{https://github.com/supranational/blst/blob/master/src/vect.h#L142}{\texttt{ct\_inverse\_mod\_256}}, 
\href{https://github.com/supranational/blst/blob/415d4f0e2347a794091836a3065206edfd9c72f3/src/recip.c#L131}{\texttt{redc\_mont\_256}}, and 
\href{https://github.com/supranational/blst/blob/415d4f0e2347a794091836a3065206edfd9c72f3/src/recip.c#L132}{\texttt{mul\_mont\_sparse\_256}}.

Finally, while both the Python spec and the c-kzg implementation have a function named \texttt{reverse\_bits}, the c-kzg function does not 
match the spec. The spec for \texttt{reverse\_bits} defines the computation as reversing the bits of a value for some specified bit length. The c-kzg 
function defines the computation as reversing all the bits of the value as defined by the size of the memory representation (in this case a 64-bit integer). The 
c-kzg implementation has a separate function, \texttt{reverse\_bits\_limited} that does match the Python spec. It should be noted that this spec 
equivalent function is not used in the implementation of \texttt{bit\_reversal\_permutation}. Instead \texttt{bit\_reversal\_permutation} calls the 
non-equivalent \texttt{reverse\_bits} function, but it handles the result in a way that conforms to the Python spec, so that the overall computation 
is equivalent.

\clearpage
\begin{table}

\textbf{
    \caption{Comparing c-kzg vs. Python KZG Specification}
    \label{table:c-kzg-vs-python}
}
\begin{center}
    \begin{tabular}{ |c|c|c| }
        \hline
        \textbf{Category} & \textbf{Python KZG Specification} & \textbf{c-kzg implementation} \\ 
        \hline
        \multirow{3}{10em}{\href{https://github.com/ethereum/consensus-specs/blob/dev/specs/deneb/polynomial-commitments.md\#bit-reversal-permutation}{Bit-reversal permutation}}
            & \href{https://github.com/ethereum/consensus-specs/blob/dev/specs/deneb/polynomial-commitments.md\#is_power_of_two}{is\_power\_of\_two}
            & \href{https://github.com/ethereum/c-kzg-4844/blob/main/src/common/utils.c\#L35}{is\_power\_of\_two} \\
            & \href{https://github.com/ethereum/consensus-specs/blob/dev/specs/deneb/polynomial-commitments.md\#reverse_bits}{reverse\_bits} 
            & \href{https://github.com/ethereum/c-kzg-4844/blob/main/src/common/utils.c\#L64}{reverse\_bits} \\
            & \href{https://github.com/ethereum/consensus-specs/blob/dev/specs/deneb/polynomial-commitments.md\#bit_reversal_permutation}{bit\_reversal\_permutation}
            & \href{https://github.com/ethereum/c-kzg-4844/blob/main/src/common/utils.c\#L102}{bit\_reversal\_permutation} \\
        \hline
        \multirow{14}{10em}{\href{https://github.com/ethereum/consensus-specs/blob/dev/specs/deneb/polynomial-commitments.md\#bls12-381-helpers}{BLS12-381 helpers}}
            & \href{https://github.com/ethereum/consensus-specs/blob/dev/specs/deneb/polynomial-commitments.md\#multi_exp}{multi\_exp}
            & \href{https://github.com/ethereum/c-kzg-4844/blob/main/src/common/ec.c\#L42}{g1\_mul}
            , \href{https://github.com/ethereum/c-kzg-4844/blob/main/src/eip4844/eip4844.c\#L124}{g2\_mul}\\
            & \href{https://github.com/ethereum/consensus-specs/blob/dev/specs/deneb/polynomial-commitments.md\#hash_to_bls_field}{hash\_to\_bls\_field}
            & \href{https://github.com/ethereum/c-kzg-4844/blob/main/src/common/bytes.c\#L123}{hash\_to\_bls\_field}\\
            & \href{https://github.com/ethereum/consensus-specs/blob/dev/specs/deneb/polynomial-commitments.md\#bytes_to_bls_field}{bytes\_to\_bls\_field}
            & \href{https://github.com/ethereum/c-kzg-4844/blob/main/src/common/bytes.c\#L64}{bytes\_to\_bls\_field}\\
            & \href{https://github.com/ethereum/consensus-specs/blob/dev/specs/deneb/polynomial-commitments.md\#bls_field_to_bytes}{bls\_field\_to\_bytes}
            & \href{https://github.com/ethereum/c-kzg-4844/blob/main/src/common/bytes.c\#L52}{bytes\_from\_bls\_field}\\
            & \href{https://github.com/ethereum/consensus-specs/blob/dev/specs/deneb/polynomial-commitments.md\#validate_kzg_g1}{validate\_kzg\_g1}
            & \href{https://github.com/ethereum/c-kzg-4844/blob/main/src/common/bytes.c\#L81}{validate\_kzg\_g1}\\
            & \href{https://github.com/ethereum/consensus-specs/blob/dev/specs/deneb/polynomial-commitments.md\#bytes_to_kzg_commitment}{bytes\_to\_kzg\_commitment}
            & \href{https://github.com/ethereum/c-kzg-4844/blob/main/src/common/bytes.c\#L103}{bytes\_to\_kzg\_commitment}\\
            & \href{https://github.com/ethereum/consensus-specs/blob/dev/specs/deneb/polynomial-commitments.md\#bytes_to_kzg_proof}{bytes\_to\_kzg\_proof}
            & \href{https://github.com/ethereum/c-kzg-4844/blob/main/src/common/bytes.c\#L113}{bytes\_to\_kzg\_proof}\\
            & \href{https://github.com/ethereum/consensus-specs/blob/dev/specs/deneb/polynomial-commitments.md\#blob_to_polynomial}{blob\_to\_polynomial}
            & \href{https://github.com/ethereum/c-kzg-4844/blob/main/src/eip4844/blob.c\#L29}{blob\_to\_polynomial}\\
            & \href{https://github.com/ethereum/consensus-specs/blob/dev/specs/deneb/polynomial-commitments.md\#compute_challenge}{compute\_challenge}
            & \href{https://github.com/ethereum/c-kzg-4844/blob/main/src/eip4844/eip4844.c\#L156}{compute\_challenge}\\
            & \href{https://github.com/ethereum/consensus-specs/blob/dev/specs/deneb/polynomial-commitments.md\#bls_modular_inverse}{bls\_modular\_inverse}
            & \href{https://github.com/ethereum/c-kzg-4844/blob/main/bindings/go/blst_headers/blst.h\#L100}{blst\_fr\_inverse}
            , \href{https://github.com/ethereum/c-kzg-4844/blob/main/bindings/go/blst_headers/blst.h\#L99}{blst\_fr\_eucl\_inverse}\\
            & \href{https://github.com/ethereum/consensus-specs/blob/dev/specs/deneb/polynomial-commitments.md\#div}{div}
            & \href{https://github.com/ethereum/c-kzg-4844/blob/main/src/common/fr.c\#L75}{fr\_div}\\
            & \href{https://github.com/ethereum/consensus-specs/blob/dev/specs/deneb/polynomial-commitments.md\#g1_lincomb}{g1\_lincomb}
            & \href{https://github.com/ethereum/c-kzg-4844/blob/main/src/common/lincomb.c\#L30}{g1\_lincomb\_naive}
            , \href{https://github.com/ethereum/c-kzg-4844/blob/main/src/common/lincomb.c\#L64}{g1\_lincomb\_fast}\\
            & \href{https://github.com/ethereum/consensus-specs/blob/dev/specs/deneb/polynomial-commitments.md\#compute_powers}{compute\_powers}
            & \href{https://github.com/ethereum/c-kzg-4844/blob/main/src/common/utils.c\#L143}{compute\_powers}\\
            & \href{https://github.com/ethereum/consensus-specs/blob/dev/specs/deneb/polynomial-commitments.md\#compute_roots_of_unity}{compute\_roots\_of\_unity}
            & \href{https://github.com/ethereum/c-kzg-4844/blob/main/src/setup/setup.c\#L93}{compute\_roots\_of\_unity}\\
        \hline
        \multirow{2}{10em}{\href{https://github.com/ethereum/consensus-specs/blob/dev/specs/deneb/polynomial-commitments.md\#polynomials}{Polynomials}}
            & \multirow{2}{10em}{\href{https://github.com/ethereum/consensus-specs/blob/dev/specs/deneb/polynomial-commitments.md\#evaluate_polynomial_in_evaluation_form}{evaluate\_polynomial\\\_in\_evaluation\_form}}
            & \multirow{2}{10em}{\href{https://github.com/ethereum/c-kzg-4844/blob/main/src/eip4844/eip4844.c\#L201}{evaluate\_polynomial\\\_in\_evaluation\_form}}\\
            & & \\ %blank so that we can get a second row used by the really long function names above
        \hline
        \multirow{11}{10em}{\href{https://github.com/ethereum/consensus-specs/blob/dev/specs/deneb/polynomial-commitments.md\#kzg}{KZG}}
            & \href{https://github.com/ethereum/consensus-specs/blob/dev/specs/deneb/polynomial-commitments.md\#blob_to_kzg_commitment}{blob\_to\_kzg\_commitment}
            & \href{https://github.com/ethereum/c-kzg-4844/blob/main/src/eip4844/eip4844.c\#L275}{blob\_to\_kzg\_commitment}\\
            & \href{https://github.com/ethereum/consensus-specs/blob/dev/specs/deneb/polynomial-commitments.md\#verify_kzg_proof}{verify\_kzg\_proof}
            & \href{https://github.com/ethereum/c-kzg-4844/blob/main/src/eip4844/eip4844.c\#L308}{verify\_kzg\_proof}\\
            & \href{https://github.com/ethereum/consensus-specs/blob/dev/specs/deneb/polynomial-commitments.md\#verify_kzg_proof_impl}{verify\_kzg\_proof\_impl}
            & \href{https://github.com/ethereum/c-kzg-4844/blob/main/src/eip4844/eip4844.c\#L349}{verify\_kzg\_proof\_impl}\\
            & \href{https://github.com/ethereum/consensus-specs/blob/dev/specs/deneb/polynomial-commitments.md\#verify_kzg_proof_batch}{verify\_kzg\_proof\_batch}
            & \href{https://github.com/ethereum/c-kzg-4844/blob/main/src/eip4844/eip4844.c\#L674}{verify\_kzg\_proof\_batch}\\
            & \href{https://github.com/ethereum/consensus-specs/blob/dev/specs/deneb/polynomial-commitments.md\#compute_kzg_proof}{compute\_kzg\_proof}
            & \href{https://github.com/ethereum/c-kzg-4844/blob/main/src/eip4844/eip4844.c\#L392}{compute\_kzg\_proof}\\
            & \multirow{2}{10em}{\href{https://github.com/ethereum/consensus-specs/blob/dev/specs/deneb/polynomial-commitments.md\#compute_quotient_eval_within_domain}{compute\_quotient\\\_eval\_within\_domain}}
            & \multirow{2}{10em}{\href{https://github.com/ethereum/c-kzg-4844/blob/main/src/eip4844/eip4844.c\#L469}{compute\_kzg\_proof\_impl}}\\
            & & \\ %blank so that we can get a second row used by the really long function names above
            & \href{https://github.com/ethereum/consensus-specs/blob/dev/specs/deneb/polynomial-commitments.md\#compute_kzg_proof_impl}{compute\_kzg\_proof\_impl}
            & \href{https://github.com/ethereum/c-kzg-4844/blob/main/src/eip4844/eip4844.c\#L424}{compute\_kzg\_proof\_impl}\\
            & \href{https://github.com/ethereum/consensus-specs/blob/dev/specs/deneb/polynomial-commitments.md\#compute_blob_kzg_proof}{compute\_blob\_kzg\_proof}
            & \href{https://github.com/ethereum/c-kzg-4844/blob/main/src/eip4844/eip4844.c\#L516}{compute\_blob\_kzg\_proof}\\
            & \href{https://github.com/ethereum/consensus-specs/blob/dev/specs/deneb/polynomial-commitments.md\#verify_blob_kzg_proof}{verify\_blob\_kzg\_proof}
            & \href{https://github.com/ethereum/c-kzg-4844/blob/main/src/eip4844/eip4844.c\#L551}{verify\_blob\_kzg\_proof}\\
            & \href{https://github.com/ethereum/consensus-specs/blob/dev/specs/deneb/polynomial-commitments.md\#verify_blob_kzg_proof_batch}{verify\_blob\_kzg\_proof\_batch}
            & \href{https://github.com/ethereum/c-kzg-4844/blob/main/src/eip4844/eip4844.c\#L752}{verify\_blob\_kzg\_proof\_batch}\\
        \hline
    \end{tabular}
\end{center}
\end{table}

\clearpage

\section{Next Steps}
The next phase of this grant's work is to create a Cryptol specification that matches the Python KZG Specification.

After this second milestone, we will create a test bench that will verify the Cryptol specification against the Python KZG Specification, 
using the Python as the oracle for property based testing of the Cryptol. We will also verify properties for KZG correct construction,
e.g. calling \texttt{compute\_kzg\_proof} with valid inputs for \texttt{blob} and \texttt{z\_bytes} always results in successful 
verification when the output proof is passed to \texttt{verify\_kzg\_proof}.

After this third milestone, we will create a script using the Software Analysis Workbench (SAW) to attempt to formally verify and 
prove one or more of the Python KZG Specification functions as implemented in c-kzg is functionally equivalent to the corresponding Cryptol 
specification function, as well as memory safe.

After this fourth milestone, we will deliver a final report with our results and conclusions.


\end{document}
